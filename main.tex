\documentclass[12pt,a4paper]{article}

% Packages
\usepackage[utf8]{inputenc}   % Encoding
\usepackage[T1]{fontenc}
\usepackage{lipsum}           % Dummy text
\usepackage{geometry}         % Page layout
\usepackage{graphicx}         % Images
\usepackage{amsmath, amssymb} % Math symbols
\usepackage{hyperref}         % Clickable links
\usepackage{setspace}         % Line spacing

% Page setup
\geometry{margin=1in}
\setstretch{1.25}

% Title info
\title{Workshop Proposal: The Doctoral Program of the International Conference on Principles and Practice of Constraint Programming (CP) 2026}
\author{Mohamed Siala}
\date{\today}

\begin{document}

\maketitle

%{A short scientific justification of the proposed topic, 
%its significance, and the particular benefits of the workshop to one of
% the scientific communities using logic in computer science,
% as well as a list of previous or related workshops (if relevant).}

\section{Doctoral Program Overview}

The Doctoral Program of the CP conference is an annual scientific event taking place during
the CP conference.
It aims at helping students working on different aspects of Constraint Programming to
socialise, network, and get in touch with experts in the field.


Submissions are of two types:
\begin{enumerate}
	\item Students with accepted papers at the CP conference may submit a two-page abstract.
	\item Otherwise, a short paper of up to 8 pages (excluding references) must be submitted.
\end{enumerate}


%https://www.floc26.org/call-for-workshops

\begin{itemize}
	\item \textbf{Scientific justification of the proposed topic, its significance, and the particular benefits of the workshop to one of the scientific communities using logic in computer science, as well as a list of previous or related workshops (if relevant):}

\end{itemize}


\section{Organisation}

\begin{itemize}
	\item \textbf{Contact information}: Mohamed Siala, Associate Professor, LAAS-CNRS \& INSA Toulouse
	\item \textbf{Proposed affiliated conference}: CP
	\item \textbf{Estimate of the number of workshop participants}: 30
	\item \textbf{Proposed format and agenda}:

	      The DP will include paper presentations,
	      posters, and two invited talks.
	      We aim to feature speakers from both
	      academia and industry.
	      In line with the Doctoral Program (DP) tradition,
	      every student will be assigned a mentor.
	      A social dinner is also planned to enhance socialising and networking upon availability of sponsorship?


	\item \textbf{Potential invited speakers}:
	      \begin{itemize}
		      \item Carla Gomes, Professor, Cornell
		      \item Nina Narodytska, Researcher, VMware
		      \item Serdar Kadioglu, VP of AI, Fidelity Investments
		            %\item Francesca Rossi, IBM Fellow
		      \item Christine Solnon, Professor, INSA Lyon
		      \item Claude-Guy Quimper, Professor, Université Laval
		      \item Pierre Schaus, Professor, UCLouvain

	      \end{itemize}

	\item \textbf{Procedures for selecting papers and participants}:
	      All papers will undergo a double-blind review process.
	      In line with DP tradition, every student will take part in the reviewing process.
	      Additionally, each paper will be reviewed by at least one senior researcher in CP.

	      %As the DP aims to develop academic maturity of young CP researchers, 
	      %all reasonable submissions will be accepted.

	\item \textbf{Plans for dissemination}: The papers will be made available at the DP website.
	\item Depending on the number of submission, the duration can be one or two days.
	\item \textbf{Preferred period}: First block (July 18-19).

\end{itemize}




\end{document}
