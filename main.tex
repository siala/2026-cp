\documentclass[12pt,a4paper]{article}

% Packages
\usepackage[utf8]{inputenc}   % Encoding
\usepackage[T1]{fontenc}
\usepackage{lipsum}           % Dummy text
\usepackage{geometry}         % Page layout
\usepackage{graphicx}         % Images
\usepackage{amsmath, amssymb} % Math symbols
\usepackage{hyperref}         % Clickable links
\usepackage{setspace}         % Line spacing

% Page setup
\geometry{margin=1in}
\setstretch{1.25}

% Title info
\title{Workshop Proposal: Doctoral Program of the International Conference on Principles and Practice of Constraint Programming (CP) 2026}
\author{Mohamed Siala}
\date{\today}

\begin{document}

\maketitle

% https://www.floc26.org/call-for-workshops

\section{Overview}

The Doctoral Program (DP) of the
CP conference is an annual scientific
event that has been held for more than two decades.
It aims to help students working
on different aspects of Constraint Programming
to socialise, network, and engage
with experts in the field.

Submissions are of two types:
\begin{enumerate}
	\item Students with accepted papers at
	      the CP conference may submit a two-page abstract.
	\item Otherwise, a short paper of up to
	      eight pages (excluding references) must be submitted.
\end{enumerate}

The DP offers a dedicated mentoring
experience in which each student
is paired with a mentor,
who may be an early-career or senior researcher.
This structure provides students
with valuable opportunities
to receive guidance, gain insights from experts in the field,
and strengthen their career trajectory.

\section{Organisation}

\begin{itemize}
	\item \textbf{Contact information}: Mohamed Siala, Associate Professor, LAAS-CNRS \& INSA Toulouse
	\item \textbf{Proposed affiliated conference}: CP 2026
	\item \textbf{Estimated number of participants}: 30
	\item \textbf{Proposed format and agenda}: \\
	      The DP will include paper presentations,
	      posters, and two invited talks.
	      We aim to feature
	      speakers from both
	      academia and industry.
	      In line with DP tradition,
	      every student will be assigned a mentor.
	      A social dinner
	      is also planned to enhance socialising and networking,
	      subject to the availability of sponsorship.

	\item \textbf{Potential invited speakers}:
	      \begin{itemize}
		      \item Carla Gomes, Professor, Cornell University
		      \item Nina Narodytska, Researcher, VMware
		      \item Serdar Kadioglu,
		            VP of AI, Fidelity Investments
		            % \item Francesca Rossi, IBM Fellow
		      \item Christine Solnon, Professor, INSA Lyon
		      \item Claude-Guy Quimper, Professor, Université Laval
		            %   \item Pierre Schaus, Professor, UCLouvain
	      \end{itemize}

	\item \textbf{Procedures for selecting papers and participants}: \\
	      All papers will undergo a double-blind review process.
	      In line with DP
	      tradition, every student will take part in the reviewing process.
	      Additionally, each paper will be reviewed by at least one senior researcher in CP.

	\item \textbf{Plans for dissemination}: \\
	      The accepted papers will be made available on the DP website.

	\item \textbf{Duration}: Depending on the number of submissions,
	      the DP may run for one or two days.

	\item \textbf{Preferred period}: First block (July 18--19).
\end{itemize}

\end{document}
